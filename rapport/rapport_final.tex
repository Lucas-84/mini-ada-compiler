\documentclass[a4paper,12pt]{article}
\usepackage[utf8]{inputenc}
\usepackage[T1]{fontenc}
\usepackage{listings}
\usepackage[left=3cm,right=3cm,top=3cm,bottom=3cm]{geometry}
\title{Rapport sur la deuxième partie du projet de compilation}
\author{Lucas \textsc{Pesenti}}
\date{}

\usepackage{listings}
\usepackage{color}
\usepackage{array}

\definecolor{mygreen}{rgb}{0,0.6,0}
\definecolor{mygray}{rgb}{0.5,0.5,0.5}
\definecolor{mymauve}{rgb}{0.58,0,0.82}

\lstset{ %
  backgroundcolor=\color{white},   % choose the background color; you must add \usepackage{color} or \usepackage{xcolor}; should come as last argument
  basicstyle=\ttfamily\footnotesize,        % the size of the fonts that are used for the code
  breakatwhitespace=false,         % sets if automatic breaks should only happen at whitespace
  breaklines=true,                 % sets automatic line breaking
  captionpos=b,                    % sets the caption-position to bottom
  commentstyle=\color{mygreen},    % comment style
  deletekeywords={...},            % if you want to delete keywords from the given language
  escapeinside={\%*}{*)},          % if you want to add LaTeX within your code
  extendedchars=true,              % lets you use non-ASCII characters; for 8-bits encodings only, does not work with UTF-8
  frame=single,	                   % adds a frame around the code
  keepspaces=true,                 % keeps spaces in text, useful for keeping indentation of code (possibly needs columns=flexible)
  keywordstyle=\color{blue},       % keyword style
  language=caml,                   % the language of the code
  morekeywords={*,...},            % if you want to add more keywords to the set
  numbers=left,                    % where to put the line-numbers; possible values are (none, left, right)
  numbersep=5pt,                   % how far the line-numbers are from the code
  numberstyle=\tiny\color{mygray}, % the style that is used for the line-numbers
  rulecolor=\color{black},         % if not set, the frame-color may be changed on line-breaks within not-black text (e.g. comments (green here))
  showspaces=false,                % show spaces everywhere adding particular underscores; it overrides 'showstringspaces'
  showstringspaces=false,          % underline spaces within strings only
  showtabs=false,                  % show tabs within strings adding particular underscores
  stepnumber=1,                    % the step between two line-numbers. If it's 1, each line will be numbered
  stringstyle=\color{mymauve},     % string literal style
  tabsize=2,	                   % sets default tabsize to 2 spaces
  title=\lstname                   % show the filename of files included with \lstinputlisting; also try caption instead of title
}

\begin{document}

\maketitle
\paragraph*{}
On ajoute à la partie précédente le fichier \texttt{compiler.ml}, réalisant la production de code. Il utilise le module fourni \texttt{x86\_64.ml}, auquel on
a rajouté le support de l'instruction \texttt{idivl} permettant de faire des divisions sur 32 bits.

\section{Modifications du typeur}

\section{Types accès et références}

\section{Retour des enregistrements}

\paragraph*{}
Les types enregistrements sont traités de manière analogue aux types classiques. Pour réaliser l'abstraction, on dispose d'une fonction \texttt{size\_of\_type}
associant à chaque type sa taille en octets. Alors, toutes les opérations de copie sont réalisées, quel que soit le type sous-jacent en itérant une opération
de copie élémentaires de 8 octets autant de fois que nécessaire. Un problème se pose cependant lorsque l'on veut retourner un enregistrement, le registre
\texttt{\%rax} ne suffit alors plus. On adopte en conséquence la convention suivante : toute fonction retournant une valeur d'un certain type la place dans la pile,
juste avant ses arguments. Ainsi, le modèle d'un tableau d'activation pour une fonction devient :

\begin{center}
\begin{tabular}{|c|}
  \hline
  Valeur de retour\\
  \hline
  Argument 1\\
  \hline
  \vdots\\
  \hline
  Argument $n$\\
  \hline
  \texttt{\%rbp} de la procédure père\\
  \hline
  Adresse de retour\\ 
  \hline
  \texttt{\%rbp} appelant\\
  \hline
  Variables locales\\
  \hline
\end{tabular}

\vdots
\end{center}

Pour les procédures, on garde le même tableau d'activation, en enlevant l'espace alloué pour la valeur de retour.

\section{Allocation dynamique}

\paragraph*{}
La construction \texttt{new} de Mini-Ada est compilée en appelant \texttt{malloc}, puis en initialisant la mémoire allouée à 0\footnote{Par exemple, le fichier
test \texttt{bst.adb} nécessite une telle initialisation.}, un peu dans la même idée que la fonction \texttt{calloc} du C.

%\begin{figure}
%\caption{Compilation de \texttt{new}}
%\label{opnew}
\begin{lstlisting}
let sz = size_of_type env (Trecord i) in
movq (imm sz) (reg rdi) ++
call "malloc" ++
(* Initialize to 0 *)
movq (reg rax) (reg rdi) ++
iter (fun () -> movq (imm 0) (ind rdi) ++
                subq (imm wordsz) (reg rdi))
     (sz / wordsz) () ++
pushq (reg rax)
\end{lstlisting}
%\end{figure}

\section{Opérations sur 32 bits}

\paragraph*{}
Une attention particulière a été portée à la gestion des opérations arithmétiques. En particulier, il fallait travailler sur 32 bits, alors que les données
sont stockées sur 64 bits. Dans cette optique, on utilise les instructions suffixées par \texttt{l}, puis, pour récupérer une valeur sur 64 bits quand on
veut l'empiler, on utilise l'instruction \texttt{movslq} permettant de faire la conversion en étendant correctement les signes. Par exemple, l'opération
\texttt{+} est compilée de la manière suivante : 

%\begin{figure}
%\caption{Compilation de l'opération {+}}
%\label{opplus}
\begin{lstlisting}
compile_expr env e1 ++ compile_expr env e2 ++
popq rcx ++ popq rax ++
addl (reg ecx) (reg eax)
movslq (reg eax) rax ++
pushq (reg rax)
\end{lstlisting}
%\end{figure}

\end{document}
